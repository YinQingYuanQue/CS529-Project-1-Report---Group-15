\documentclass{article}
\usepackage{graphicx} % Required for inserting images
\usepackage[ruled,vlined]{algorithm2e}
\usepackage{xcolor} % for colored text
\usepackage{graphicx} % for including graphics
\usepackage{amsmath} % for advanced math features

\title{CS529 Project 1 Report - Group 15}
\vspace{3cm}
\author{%
    \begin{tabular}[t]{c@{\hskip 2em}c} % Two columns with some space in between
        Behnoud Alaghband & Zhuoming Liu  \\
        University of New Mexico & University of New Mexico \\
        \texttt{balaghband@unm.edu} & \texttt{dawnmoon@unm.edu}
    \end{tabular}
}


\begin{document}

\maketitle

\section{Discussion on the different options for split criteria}


\subsection{Categorical Data}
\begin{verbatim}
['ProductCD', 'card1', 'card2', 'card3', 'card4', 'card5', 'card6', 'addr1', 'addr2']
\end{verbatim}

\subsection{Numerical Data}
\begin{verbatim}
['C1', 'C10', 'C11', 'C12', 'C13', 'C14', 'C2', 'C3', 'C4', 'C5', 'C6', 'C7', 
'C8', 'C9', 'TransactionAmt', 'TransactionDT', 'isFraud']
\end{verbatim}

we calculate information gain by Info-Gain function :

\begin{algorithm}[H]
\SetAlgoLined
\KwIn{Input\_Dataframe, Split\_feature, IG\_methods, Num\_Cat}
\KwOut{Information Gain}
\KwResult{Calculate Information Gain based on the given parameters}

\SetKwFunction{InfoGain}{Info\_Gain}
\SetKwProg{Fn}{Function}{:}{}

\Fn{\InfoGain{Input\_Dataframe, Split\_feature, IG\_methods, Num\_Cat}}{
    % Function body
}

\caption{Info\_Gain function}
\end{algorithm}

\textbf{Information gain function}

\begin{center}
\begin{tabular}{|c|c|p{7cm}|}
\hline
\textbf{Input Variable} & \textbf{Type} & \textbf{Definition} \\
\hline
\textcolor{green}{Input\_Dataframe} & \texttt{pandas.DataFrame} & This is your chosen subset of training data \\
\textcolor{green}{Split\_feature} & \texttt{String} & The chosen feature for calculating Information gain \\
\textcolor{green}{IG\_methods} & \texttt{String} & Three options: \textcolor{red}{Gini} for Gini index, \textcolor{red}{Entropy} for Max Entropy, \textcolor{red}{MisEr} for Miss classification\\
\textcolor{green}{Num\_Cat} & \texttt{String} & Two options: \textcolor{red}{Num} for a numeric feature, \textcolor{red}{Cat} for a categorical feature \\
\hline
\end{tabular}
\end{center}

This function has two possible returns: 

\begin{itemize}
    \item \textcolor{red}{[IG\_feature, Split point]}: if \textcolor{green}{Split\_feature} is a \textbf{numeric} feature
        \begin{itemize}
            \item \textbf{IG\_feature}: single float variable representing \textbf{Information Gain} of \textcolor{green}{Split\_feature}
            \item \textbf{Split point}: single float variable, the best binary split point for this numeric feature \textcolor{green}{Split\_feature}
        \end{itemize}
    \item \textcolor{red}{[IG\_feature, -3.14]}: if \textcolor{green}{Split\_feature} is a \textbf{categorical} feature
        \begin{itemize}
            \item \textbf{IG\_feature}: same as previous
            \item \textbf{-3.14}: Meaningless, and not to be used in the future. Just to make this function work recursively.
        \end{itemize}
\end{itemize}


\section{Discussion on the different values for $\alpha$ in chi square}

$\alpha$ represents the significance level as probability that a hypothesis is rejected when it is actually true. The common value for $\alpha$ is 0.05 which means if the probability of the value is less than 5\%, the null hypothesis is rejected and there is a statistical significance between the data which has been observed and frequencies which are expected.



\section{Discussion wrt design decisions for: missing data, numerical features, class imbalance}
\section{Interpreting the results and provides insights into the various model performance}
\section{Insights into the relationship between features and the target variable}


\end{document}
